\documentclass[twoside,pl,final]{labman}

\usepackage{graphicx}
\usepackage{float}
\usepackage{url}
\usepackage{listings}
\usepackage[caption=false]{subfig}
\usepackage{placeins}

\graphicspath{ {fig/} }

\subject{Projektowanie układów analogowych dla systemów VLSI}
\title{Lustra prądowe i układ polaryzacji}
\author{mgr inż. Jakub Kopański\\
dr inż. Tomasz Borejko}

\begin{document}
\maketitle
\tableofcontents
\clearpage
\listoffigures
\clearpage
\listoftables
\clearpage

\chapter{Wstęp}
\label{intro}
Najważniejszym zagadnieniem przy projektowaniu
analogowych układów scalonych jest polaryzacja.
Wybór i zapewnienie odpowiedniego punktu pracy
ma wpływ na szybkość działania układu,
dopasowanie elemetnów, zakres pracy,
odporność na zakłócenia zasilania i masy
oraz na moc zużywaną przez układ.

W analogowych układach scalonych,
ze względu na łatwość dopasowania elementów,
do polaryzacji tranzystorów wykorzystuję się źródła/lustra prądowe.

W ninejszym ćwiczeniu studenci zostaną zapoznani
z projektowaniem luster prądowych
a zdobyta wiedza posłuży do stworzenia układu polaryzacji,
który zostanie wykorzystany przy kolejnym ćwiczeniu.

\chapter{Lustra prądowe}
\label{mirror}

\section{Podstawowe lustro prądowe}
\label{mirror:basic}
\begin{figure}[!htbp]
  \centering
  \includegraphics[width=0.9\textwidth]{mirror_basic}
  \caption{Podstawowe lustro prądowe}
  \label{fig:mirror:basic}
\end{figure}

Podstawowe lustro prądowe,
wykonanae z tranzystorów typu~\emph{N},
pokazano na~\fig{fig:mirror:basic}
%Przyjmijmy, że oba tranzystory mają takie same wymiary.
Z topologi układu wynika, że~$V_{GS1} = V_{DS1} = V_{GS2}$.
Pomijając wpływ modulacji długości kanału (parametr~$\lambda$),
dzięki równości napięć bramka - źródło obu tranzystorów spodziewamy się,
że oba transytory będą miały jednakowy prąd drenu.
Jeżeli oba rezystory mają taką samą wartość rezystancji,
potencjały drenu obu tranzystorów są takie same.
Dopasowując wymiary, napięcia~$V_{GS}$ i prądy drenu~$I_D$ obu tranzystorów,
możemy być pewni, że napięcia dren - źródło obu tranzystorów są jednakowe
($V_{GS1} = V_{DS1} = V_{GS2} = V_{DS2}$).

\begin{figure}[!htbp]
  \centering
  \includegraphics[width=0.9\textwidth]{mirror_equivalent}
  \caption{Lustro prądowe i schemat zastępczy}
  \label{fig:mirror:basic:equivalent}
\end{figure}

Na~\fig{fig:mirror:basic:equivalent} zaprezentowano jak możemy myśleć
o \emph{wyjściu} lustra prądowego.

\begin{equation}
  I_{REF} = I_{D1} = \frac{K_n}{2} \cdot \frac{W_1}{L_1} \cdot (V_{GS1} - V_{TH}) ^ 2 (1 + \lambda V_{DS1})
  \label{eq:mirror:basic:iref}
\end{equation}
Ponieważ~$V_{DS1} = V_{GS1}$ i~$V_{DS1sat} = V_{GS1} - V_{TH}$,
dla tranzystora~$M_2$ możemy napisać:
\begin{equation}
  I_O = I_{D2} =  \frac{K_n}{2} \cdot \frac{W_2}{L_2} \cdot (V_{GS2} - V_{TH}) ^ 2 (1 + \lambda V_O)
  \label{eq:mirror:basic:iout}
\end{equation}
Napięcie na źródle prądowym oznaczmy~$V_O$.
Jak zostało zauważone na wstępie tego rozdziału:~$V_{GS1} = V_{GS2}$,
dzięki temu stosunek prądów drenu tranzystorów ma postać:
\begin{equation}
  \frac{I_O}{I_{REF}} = \frac{W_2 / L_2}{W_1 / L_1} \cdot \frac{1 + \lambda V_{O}}{1 + \lambda V_{DS1}}
  \label{eq:mirror:basic:iratio}
\end{equation}
Długości kanałów tranzystorów luster prądowych są takie same.
Pomijając na razie wpływ modulacji długości kanału,
możemy zapisać:
\begin{equation}
  \frac{I_O}{I_{REF}} = \frac{W_2}{W_1}
  \label{eq:mirror:basic:wratio}
\end{equation}
Poprzez proste skalowanie szerokości kanału,
możemy zmieniać prąd wyjściowy lustra.
Przykładowe zastosowanie pokazano na~\fig{fig:mirror:basic:scale}

\begin{figure}[!htbp]
  \centering
  \includegraphics[width=0.9\textwidth]{mirror_scale}
  \caption{Skalowanie prądu luster}
  \label{fig:mirror:basic:scale}
\end{figure}

\subsection{Dopasowywanie prądów}
\label{mirror:basic:matching}
Podstawowym problemem przy projektowaniu lustra prądowego
jest zapewnienie równości prądów referencyjnego~$I_{REF}$
oraz wyjściowego~$I_O$.
W następnych sekcjach zbadamy jak różnice pewnych parametrów
wpływają na różnice prądów.

\subsubsection{Różnica napięć progowych}
\label{mirror:basic:marching:vth}
W punkcie~\ref{mirror:basic} powiedzieliśmy,
że w pierwszym przybliżeniu,
równość prądów $I_{REF}$ oraz $I_O$ wynika z równości napięć $V_{GS}$ lustra prądowego.
Chcąc zbadać wpływ różnicy napięć progowych przyjmujemy, źe:
\begin{eqnarray}
  V_{TH1} &= V_{TH} - \frac{\Delta V_{TH}}{2} \\
  V_{TH2} &= V_{TH} + \frac{\Delta V_{TH}}{2}
\end{eqnarray}


\subsubsection{Różnica transkonduktancji}
\label{mirror:baisc:mathcing:kp}

\subsubsection{Różnica napięc $V_{DS}$}
\label{mirror:basic:matching:vds}

\section{Kaskodowe lustro prądowe}
\label{mirror:cascode}

\subsection{Prosta kaskoda}
\label{mirror:cascode:simple}

\subsection{Niskonapięciowa kaskoda}
\label{mirror:cascode:wideswing}

\subsubsection{Generowanie napięcia polaryzacji tranzystora kaskodującego}
\label{mirror:cascode:wideswing:mws}

\subsubsection{Kaskoda z wykorzystaniem tranzystorów krótko-kanałowych}
\label{mirror:cascode:wideswing:shortchannel}

\chapter{Układ polaryzacji}
\label{bias}

\bibliographystyle{IEEEtran}
\bibliography{IEEEabrv,bibliography}

\end{document}
