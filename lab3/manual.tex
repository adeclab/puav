\documentclass[twoside,pl,final]{labman}

\usepackage{graphicx}
\usepackage{float}
\usepackage{url}
\usepackage{listings}
%\usepackage[caption=false]{subfig}
\usepackage{placeins}

\graphicspath{ {fig/} }

\subject{Projektowanie układów analogowych dla systemów VLSI}
\title{Wzmacniacz operacyjny}
\author{mgr inż. Jakub Kopański}

\begin{document}
\maketitle
\tableofcontents
\clearpage
\listoffigures
\clearpage
\listoftables
\clearpage

\chapter{Wstęp}
\label{intro}
Wzmacniacze operacyjne są jednymi z podstawowych bloków wykorzystywanych do budowania bardziej złożonych układów.
Będąc tak powszechnymi elementami większych systemów, omówimy dokładnie ich działanie oraz zagadnienia związane z ich projetowaniem.
Wzmacniacz operacyjny przeznaczony jest do pracy w układzie sprzężenia zwrotnego.
Dzięki temu o parametrach układu decydują wartości elementów z których wykonano sprzężenie zwrotne,
a nie bezwzgle parametr wzmacniacza operacyjnego.
Takie rozwiązanie jest bardzo korzystne w realizacji scalonej ponieważ o parametrach układu można decydować stosunkiem wartości elementów sprzężenia.

\begin{figure}[!htbp]
  \centering
  \includegraphics[width=0.9\textwidth]{opamp_idea}
  \caption{Schemat poglądowy wzmacniacza operacyjnego.}
  \label{fig:opamp:idea}
\end{figure}

Ideowy schemat wzmacniacza operacyjnego zaprezentowano na~\fig{fig:opamp:idea}
Składa się on ze wzmacniacza różnicowego, stopnia wzmacniającego oraz bufora wyjściowego.
Dzięki wzmacniaczowi różnicowemy mamy~2 wejścia wzmacniacza
(do jednego będziemy podawać sygnał sprzężenia zwrotnego).
Stopień wzmacniający zapewnia odpowiednio wysokie wzmocnienie całgo toru.
Często przy wykorzystaniu stopnia wzmacniającego kompensuje się charakterystyke częstotliwościową wzmacniacza operacyjnego,
o czym powiemy w dalszej części instrukcji.
Ostatnim elementem jest bufor wyjściowy.
Zapewnia on małą rezystancje wyjściową wzmacniacza operacyjnego,
przez co możliwe jest obciążanie wzmacniacza \emph{małymi rezystancjami}.

W układach scalonych rzadko istnieje konieczność wysterowania obciążenia rezystancyjnego.
Typowo wzmacniacze operacyjne muszą wysterować obciążenie o charakterze pojemnościowym - bramkę tranzystora.
W takim przypadku można zrezygnować z bufra wyjściowego.
Tego typu wzmacniacz często nazywa się wzmacniaczem transkonduktancyjnym~\emph{OTA}
\eng{Operational Transconductance Amplifier}.
Rezygnując z bufora wyjściowego, należy zwrócić także uwagę na ewentualne problemy ze \emph{slew rate}
w szczególności gdy wartość pojemności obciążająca wzmacniacz jest duża.
Ze względu na powszechność z jaką używa się wzmacniaczy~\emph{OTA} w układach scalonych często nazywa się je wzmacniaczami operacyjnymi.

\chapter{Projektowanie wzmacniacza operacyjnego}
\label{opamp}

\section{Schemat elektryczny projektowanego układu}
\label{opamp:schematic}
\begin{figure}[!htbp]
  \centering
  \includegraphics[width=0.9\textwidth]{opamp_sch}
  \caption{Schemat elektryczny wzmacniacza operacyjnego.}
  \label{fig:opamp:sch}
\end{figure}

Schemat układu jakim będziemy projektować zaprezentowano na~\fig{fig:opamp:sch}
Składa się ze wzmacniacza różnicowego który steruję wzmacniaczem o wspólnym źródle.
Wzmocnienie układu w otwartej pętli,
dla niskich częstotliwości jest iloczynem wzmocnień poszczególnych stopni i wynosi:
\begin{equation}
  A_{OLDC} = A_1 \times A_2 = \overbrace{g_{mn} \cdot (r_{dsn} || r_{dsp})}^{A_1} \times \overbrace{g_{mp} \cdot r_{dsp}}^{A_2}y
\end{equation}

\subsection{Punkt pracy}
\label{opamp:schematic:op}
Węszły \emph{bias<3>} i~\emph{bias<4>} pochodzą z bloku projektowanego na poprzednich zajęciach.
Pary traansystorów~\emph{M6} i~\emph{M8} tworzą źródła prądowe które wymuszają przepływ prądu w gałęziach układu.
Przez tranzystory \emph{M1}~i~\emph{M2} płynie taki sam prąd,
równy połowie prądu lustra złożnonego z tranzystorów~\emph{M6}.
Bramki tranzystorów~\emph{M3} i~\emph{M4} są zwarte, więc ich napięcia~$V_{GS}$ są również takie same.
Ponieważ prąd płynący przez oba tranzystory jest taki sam,
napięcia~$V_{DS}$ obu tranzystorów muszą być takie same.
Dlatego napięcie~$V_{GS}$ tranzystora~\emph{M7}, składającego się na wzmacniacz drugiego stopnia,
jest równe napięciu~$V_{GS}$ tranzystorów~\emph{M3} i~\emph{M4}.
Takie połączenie zapewni dobrze ustalony punkt pracy wzmacniacza.
Dzięki temu tranzystory będą posiadały znane parametry i możliwe będzie przewidzenie osiągów projektowanego wzmacniacza operacyjnego.

\subsection{Wejściowe napięcie wspólne}
\label{opamp:schematic:cm}
Gdy wzmacniacz operacyjny pracuję przy zamkniętej pętli sprzężenia zwrotnego,
napięcia na wejściach pary różnicowej są utrzymywane na tych samych (lub prawie tych samych) wartościach.
Wartość średnia z napięć na obu wejściach wzmacniacza nazywana jest napięciem wspólnym~\eng{common-mode voltage}.
Należy zastanowić się nad maksymalnym~$V_{CMMAX}$ i minimalnym~$V_{CMMIN}$ napięciem wspólnym które zapewni,
że tranzystory wzmacniacza różnicowego pozostaną w nasyceniu.

Aby tranzystory źródła prądowego pozostały w nasyceniu niezbędne jest napięcie co najmniej~$2V_{DSsat}$.
Stąd minimalne napięcie wspólne wynosi:
\begin{equation}
  V_{CMMIN} = 2V_{DSsatn} + V_{GSn}
\end{equation}

Górny limit napiecia wspólnego można obliczyć zauważając,
że napięcie na drenie~\emph{M2} i~\emph{M1} jest równe i wynosi~$V_{DD} - V_{SGp}$.
Dlatego możemy zapisać:
\begin{equation}
  V_{DS} \geq V_{GS} - V_{THn} \rightarrow V_D \geq V_G - V_{THN} \rightarrow V_{CMMAX} = V_{DD} - V_{SGp} + V_{THn}
\end{equation}

\subsection{Wejściowe napięcie różnicowe}
\label{opamp:schematic:diff}
Ponieważ prąd drenu tranzystors w zakresie nasycenia opisuję się równaniem:
\begin{equation}
  i_D = \frac{\beta_n}{2}(v_{GS} - V_{THN})
\end{equation}
wejściowe napięcie różnicowe można przedstawić w postacii:
\begin{equation}
  v_{DI} = \sqrt{\frac{2}{\beta_n}}(\sqrt{i_{D1}} - \sqrt{i_{D2}})
  \label{eq:opamp:schematic:diff:di}
\end{equation}
Maksymalne wejściowe napięcie różnicowe otrzymamy podstawiając prąd źródła prądowego~$I_{SS}$ w \emph{ogonie} pary różnicowej
za prąd~$i_{D1}$ do równania~\ref{eq:opamp:schematic:diff:di} oraz zerując prąd~$i_{D2}$.
Otrzmamy wtedy:
\begin{equation}
  v_{DIMAX} = v_{I1} - v_{I2} = \sqrt{\frac{2 \cdot L \cdot I_{SS}}{KP_n \cdot W}}
  \label{eq:opamp:schematic:diff:di:max}
\end{equation}
Minimalne różnicowe napięcie wejściowe otrzymujemy poprzed podstawienie prądu~$I_{SS}$ pod~$i_{D2}$ i wyzerowanie~$i_{D1}$.
\begin{equation}
  v_{DIMIN} = - v_{DIMAX} = - \sqrt{\frac{2 \cdot L \cdot I_{SS}}{KP_n \cdot W}}
\end{equation}

\begin{figure}[!htbp]
  \centering
  \includegraphics[width=0.9\textwidth]{opamp_di}
  \caption{Prądy drenów w zależności od napięcia wejściowego}
  \label{fig:opamp:schematic:diff:di}
\end{figure}

Rys.~\ref{fig:opamp:schematic:diff:di} przedstawia prądy drenu tranzystorów pary różnicowej w zależności od różnicowego napięcia wejściowego.

\section{Charakterystyka częstotliwościowa wzmacniacza}
\label{opamp:freq}
\begin{figure}[!htbp]
  \centering
  \includegraphics[width=0.9\textwidth]{opamp_smallsignal}
  \caption{Model małosygnałowy wzmacniacza operacyjnego.}
  \label{fig:opamp:freq:smallsignal}
\end{figure}

\begin{figure}[!htbp]
  \centering
  \includegraphics[width=0.9\textwidth]{opamp_smallsignal_miller}
  \caption{Model małosygnałowy po zastosowaniu efektu Millera.}
  \label{fig:opamp:freq:smallsignal:miller}
\end{figure}

Aby wyznaczyć charakterystykę częstotliwościową wzmacniacza posłużymy się modelem małosygnałowym widocznym na~\fig{fig:opamp:freq:smallsignal}
Korzystając z twierdzenia Millera można przenieść pojemność~$C_{gd}$ na węzeł~$1$ i~$2$.
Taki zabieg spowoduję, że w układzie będą 2 stałe czasowe związane z węzłami~$1$ i~$2$.
Schemat elektryczny odpowiadającego modelu małosygnałowego pokazany jest na~\fig{fig:opamp:freq:smallsignal:miller}
Częstotliwości graniczne związane z tymi stałymi czasowymi będą wynosić:
\begin{align}
  f_1 &= \frac{1}{2 \pi (C_{gs} + C_{gd7}(1 + |A_2|)) \cdot r_{ds2} || r_{ds4}} \label{eqn:opamp:freq:pole:low} \\
  f_2 &= \frac{1}{2 \pi (C_{gd8} + C_L + (1 + \frac{1}{|A_2|})C_{gd7}) \cdot r_{ds7} || R_{ocascn}} \label{eqn:opamp:freq:pole:high}
\end{align}

Charakterystyka częstotliwościowa układu z~\fig{fig:opamp:freq:smallsignal} ma postać:
\begin{equation}
  A_v(f) = \frac{\overbrace{g_{mn} \cdot (r_{dsn} || r_{dsp})}^{A_1} \times \overbrace{g_{mp} \cdot r_{dsp}}^{A_2}}{\Big(1 + j \frac{f}{f_1}\Big)\Big(1 + j \frac{f}{f_2}\Big)}
\end{equation}

\subsection{Zero w prawej płaszczyźnie}
\label{opamp:freq:rhp}
Niestety moddel zaproponowany w rozdziale~\ref{opamp:freq} nie jest całkowicie słuszny.
Stosując twierdzenie Millera, do wyznaczenia częstotliwości granicznych, pomijane jest zero w charakterystyce częstotliwościowej.
Obserwując~\fig{fig:opamp:freq:smallsignal} można zauważyć,
że w przypadku granicznym (dla bardzo wysokoch częstotliwości) pojemność~$C_{gd}$ zwiera wejście z wyjściem drugiego stopnia wzmacniacza.
Dlatego w celu wyznaczenia dokładniejszej charakterystyki częstotliwościowej nie możemy korzystać z twierdzenia Millera i schemtu zastępczego z~\fig{fig:opamp:freq:smallsignal:miller}

W celu uproszczenia obliczeń potraktujemy węzeł~$1$ jako wejście rozważanego układu, do wyznaczenia częstotliwości zera.
Schemat zaprezentowano na~\fig{fig:opamp:freq:rhp:schematic}
Wartości na schemacie są równe:
\begin{align}
  R_o &= r_{ds7} || R_{ocascn} \\
  C_o &= C_{gd8} + C_L
\end{align}
Suma prądów w węźle wyjsciowym wynosi:
\begin{equation}
  \frac{v_{out} - v_{in}}{1 / j \omega C_{gd7}} + \frac{v_{out}}{R_o || 1 / j \omega C_o} + g_{m2} \cdot v_{in} = 0
\end{equation}
Wyznaczjąc wzmocnienie układu otrzymujemy równanie:
\begin{equation}
  \frac{v_{out}}{v_{in}} = -g_{m2} R_o \cdot \frac{1 - j \omega \frac{C_{gd7}}{g_{m2}}}{1 + j \omega ( C_{gd7} + C_o ) R_o}
\end{equation}
skąd widzimy, że biegun jest taki sam jak w równaniu~\ref{eqn:opamp:freq:pole:high}.
Natomiast w liczniku transmitancji pojawiło się zero w prawej płaszczyźnie:
\begin{equation}
  f_z = \frac{g_{m2}}{2 \pi C_{gd7}}
\end{equation}

Zero w po prawej stronie układu współrzędnych ma taki sam wpływ na odpowiedź amplitudową jak jak zero po lewej stronie,
ale inny wpływ na odpowiedź fazową.
Zero w prawej płaszczyźnie wpływa na odpowiedź fazową tak samo jak biegun w lewej płaszczyźnie.
Ta właściwość rodzi ważne konsekwencje przy projektowaniu wzmacniaczy,
pracujących przy sprzężeniu zwrotnym (ale nie tylko!).
Wyjście wzmacniacza, przy dodatkowym przesunięciu fazy,
podane jako sprzężenie zwrotne może zmienić jego charakter i zsumować się z sygnałem wejściowym
(dodanie sprzeżenie zwrotne), powodując niestabilność wzmacniacza.

\subsection{Kompensacja wzmacniacza operacyjnego}
\label{opamp:schematic:compensation}
\begin{figure}[!htbp]
  \centering
  \includegraphics[width=0.9\textwidth]{opamp_signal}
  \caption{Schemat sygnałowy dla wzmacniacza operacyjnego}
  \label{fig:opamp:schematic:compensation:signal}
\end{figure}

Jednym z bardziej istotnych zagadnień przy projektowaniu wzmacniacza operacyjnego jest jego kompensacja.
\emph{Opamp} oblicza różnice pomiędzy napięciami na wejściu odwracającym i nieodwracającym
a następnie mnoży wynik przez bardzo dużą wartość (wzmocnienie).
Scehmat sygnałowy wzmacniacza operacyjnego przedstawia~\fig{fig:opamp:schematic:compensation:signal}
Wzmocnienie w otwartej pętli w funcki częstotliwości oznacza się~$A_{OL}(f)$.

\begin{figure}[!htbp]
  \centering
  \includegraphics[width=0.9\textwidth]{opamp_feedback}
  \caption{Przykładowe konfiguracja wzmacniacza operacyjnego ze sprzężeniem zwrotnym.}
  \label{fig:opamp:schematic:compensation:feedback}
\end{figure}
\begin{figure}[!htbp]
  \centering
  \includegraphics[width=0.9\textwidth]{opamp_feedback_signal}
  \caption{Schemat przepływu sygnału dla wzmacniacza ze sprzężeniem zwrotnym.}
  \label{fig:opamp:schematic:compensation:feedback:signal}
\end{figure}

Praktyczne zastosowania \emph{opampa} wykorzysują sprzężenie zwrotne.
Przykładowa aplikacja (wzmacniacz nieodwracający) została pokazana na~\fig{fig:opamp:schematic:compensation:feedback}
Analizując~\fig{fig:opamp:schematic:compensation:feedback:signal} możemy zapisać:
\begin{equation}
  v_{out} = A_{OL}(f) \times (v_{in} - v_{f})
  \label{eq:opamp:schematic:compensation:vout}
\end{equation}
oraz
\begin{equation}
  v_f = v_{out} \frac{R_2}{R_1 + R_2}
  \label{eq:opamp:schematic:compensation:vf}
\end{equation}
\emph{Ilość} sygnały wyjścowego jaka podawana jest z powrotem na wejście nazywana jest współczynnikiem sprzężenia~$\beta$.
\begin{equation}
  \beta = \frac{R_2}{R_1 + R_2}
  \label{eq:opamp:schematic:compensation:beta}
\end{equation}
Podstawiając równania~\ref{eq:opamp:schematic:compensation:vf} i~\ref{eq:opamp:schematic:compensation:beta}
do równania~\ref{eq:opamp:schematic:compensation:vout} i wyznaczając wzmocnienie w układzie zamkniętej pętli
otrzymujemy:
\begin{equation}
  A_{CL}(f) = \frac{v_{out}}{v_{in}}
  \label{eq:opamp:schematic:compensation:acl}
\end{equation}
Jeżeli wzmocnienie otwartej pętli jest bardzo duże~$A_{OL}(f) \rightarrow \infty$,
to wzmocnienie zamkniętej pętli w prezebtowanym układzie wzmacniacza nieodwracającego wynosi:
\begin{equation}
  A_{CL}(f) \rightarrow \frac{1}{\beta} = 1 + \frac{R_1}{R_2}
  \label{eq:opamp:schematic:compensation:acl:beta}
\end{equation}

\begin{figure}[!htbp]
  \centering
  \includegraphics[width=0.9\textwidth]{opamp_follower}
  \caption{Wtórnik na wzmacniaczu operacyjnym.}
  \label{fig:opamp:schematic:compensation:follower}
\end{figure}

Należy zwrócić uwagę, że gdy w równaniu~\ref{eq:opamp:schematic:compensation:acl}
czynnik~$\beta \cdot A_{OL}(f) = -1$ to wzmocnienie rośnie do nieskończoności.
Dokładniej ujmując, muszą być spełnione warunki:
\begin{equation}
  | \beta \cdot A_{OL}(f) | = 1
  \label{eq:opamp:schematic:compensation:gen:amp} \\
  \measuredangle \beta \cdot A_{OL}(f) = \pm 180^\circ
  \label{eq:opamp:schematic:compensation:gen:phase}
\end{equation}
Wzmocnienie zamkniętej pętli rosnące do nieskończoności oznacza,
że wzmacniacz jest niestabilny.
Najgorszy efekt (najłatwiej o niestabilny wzmacniacz) jest w przypadku dużych wartości współczynnika~$\beta$.
Taka sytuacja będzie miała miejsce gdy cały sygnał wyjściowy zostanie podany jako sygnał sprzężenia
(zakładając, że w torze sprzężenia zwrotnego nie ma transformatorów czy wzmacniaczy).
W przypadku wzmacniacza operacyjnego taka sytuacja ma miejsce w przypadku konfiguracji wtórnika,
widocznej na~\fig{eq:opamp:schematic:compensation:follower}
W celu zbadania stabilności wzmacniacza operacyjnego należy sprawdzić wzmocnienie otwartej pętli,
pry współczynniki sprzężenie równym~$1$, to znaczy:
\begin{equation}
  | A_{OL}(f) | = 1
  \label{eq:opamp:schematic:compensation:condition:amp} \\
  \measuredangle A_{OL}(f) = 180^\circ
  \label{eq:opamp:schematic:compensation:condition:phase}
\end{equation}

Warto zauważyć, że im większe wzmocnienie w zamkniętej pętli,
tym mniejsza wartość współczynnika sprzężenia~$\beta$,
tym bardziej stabilny będzie układ wzmacniacza operacyjnego ze sprzężeniem zwrotnym.
Sprżeżenie zwrotne sprawia, że układ staję się mniej (prawie w ogóle) wrażliwy na wahania
wzmocnienia \emph{opampa} w otwartej pętli~$A_{OL}$.
Odbywa się to kosztem stabilności.

W celu oszacowania chrakterystyki częstotliwościowej wzmacniacza,
posłużymy się modelem z~\fig{fig:opamp:schematic:compensation:model}
Z kolei na~\fig{fig:opamp:schematic:compensation:nodes} zaznaczono węzły które odpowiadają odpowiednim węzłom modelu.
Bazując na schemacie eleketrycznym \emph{opampa} możemy zapisać:
\begin{align}
  R_1 &= r_{dsn} || r_{dsp} \nonumber \\
  R_2 &= r_{dsp} || R_{ocasn} \nonumber \\
  g_{m1} &= g_{mn} \nonumber \\
  g_{m2} &= g_{mp} \nonumber \\
  C_1 &= C_{ds4} + C_{gd2} + C_{gs7} \nonumber \\
  C_2 &= C_L + C_{gd8} \approx C_L \nonumber
\end{align}

Biegun związany z węzłem~$1$ jest na częstotliwości równej
\begin{equation}
  f_1 \approx \frac{1}{2 \pi [(C_c + C_2) \cdot R_2 + (C_1 + C_c(1 + g_{m2}R_2)) \ cdot R_1]}
\end{equation}
co dla dużych wartości~$g_{m2}R_2$ można przybliżyć:
\begin{equation}
  f_1 \approx \frac{1}{2 \pi g_{m2}R_2R_1C_c}
\end{equation}
Biegun pochodzący od węzła wyjściowego znajduję się na częstotliwości:
\begin{equation}
  f_2 = \frac{g_{m2}C_c}{2 \pi (C_cC_1 + C_1C_2 + C_cC_2)}
\end{equation}
Zero transmitancji położone jest na częstotliwości:
\begin{equation}
  f_z = \frac{g_{m2}}{2 \pi \cdot C_c}
\end{equation}


\bibliographystyle{IEEEtran}
\bibliography{IEEEabrv,bibliography}

\end{document}
